\documentclass[a4paper]{article}

\usepackage[utf8]{inputenc}
\usepackage[T1]{fontenc}
\usepackage{lmodern}
\usepackage{ngerman}
\usepackage{fullpage}
% The data files, written on the first run.

\title{Hochleistungsrechnen: Aufgabenblatt 4}
\author{}
\date{}

\begin{document}
	\maketitle
	\textbf{Abgabe von:}
	\begin{itemize}
		\item Jonas Papmeier, ???
		\item Konstantin Schlese, Matrikel-Nr.: 5639178
		\item Thorsten Ploß, Matrikel-Nr.: 6324901
	\end{itemize}
	\section*{Aufgabe 1: Parallelisierung mit OpenMP}
	Wir haben relativ schnell(2 Stunden) die openmp Versionen der partdiff-seq implementiert.
	Wir haben 2 Tage gebraucht um mit verschiedenen openmp Einstellungen zu experementieren.
	
	\section*{Aufgabe 2: Umsetzung der Datenaufteilungen}
	\section*{Aufgabe 3: Vergleich der OpenMP-Scheduling-Algorithmen}
	\section*{Aufgabe 4: Leistungsanalyse}
	\subsection*{Messung 1}
	\subsubsection{sequential}
	\begin{center}
		\begin{tabular}{ l | c | c | r } \hline
		 Threads & Berechnungszeit & Norm des Fehlers \\ \hline
		 1 & 63.915467 & 2.929830e-07 \\ \hline
 	     \end{tabular}
	\end{center}
	\subsubsection{openmp-element}
	\begin{center}[
		\begin{tabular}{ l | c | c | r } \hline
		 Threads & Berechnungszeit & Norm des Fehlers \\ \hline
		 1 & 114.948134 & 2.929830e-07 \\ \hline
		 2 & 61.178141 & 2.929830e-07 \\ \hline
		 3 & 42.465717 & 2.537307e-07 \\ \hline
		 4 & 5 & 6 \\ \hline
		 5 & 5 & 6 \\ \hline
		 6 & 5 & 6 \\ \hline
		 7 & 5 & 6 \\ \hline
		 8 & 5 & 6 \\ \hline
		 9 & 5 & 6 \\ \hline
		 10 & 5 & 6 \\ \hline
		 11 & 5 & 6 \\ \hline
		 12 & 8 & 9 \\ \hline
	   \end{tabular}
	\end{center}
		\subsubsection{openmp-spalten}
	\begin{center}[
		\begin{tabular}{ l | c | c | r } \hline
		 1 & 62.591578 & 2.929830e-07 \\ \hline
		 2 & 32.036789 & 2.929830e-07 \\ \hline
		 3 & 21.502030 & 2.537307e-07 \\ \hline
		 4 & 5 & 6 \\ \hline
		 5 & 5 & 6 \\ \hline
		 6 & 5 & 6 \\ \hline
		 7 & 5 & 6 \\ \hline
		 8 & 5 & 6 \\ \hline
		 9 & 5 & 6 \\ \hline
		 10 & 5 & 6 \\ \hline
		 11 & 5 & 6 \\ \hline
		 12 & 8 & 9 \\ \hline
	   \end{tabular}
	\end{center}
	\subsubsection{openmp-zeilen}
	\begin{center}[
		\begin{tabular}{ l | c | c | r } \hline
		 1 & 84.126204 & 2.929830e-07 \\ \hline
		 2 & 45.604561  & 2.929830e-07 \\ \hline
		 3 & 30.055612 & 2.537307e-07 \\ \hline
		 4 & 22.858381 & 2.071703e-07 \\ \hline
		 5 & 5 & 6 \\ \hline
		 6 & 5 & 6 \\ \hline
		 7 & 5 & 6 \\ \hline
		 8 & 5 & 6 \\ \hline
		 9 & 5 & 6 \\ \hline
		 10 & 5 & 6 \\ \hline
		 11 & 5 & 6 \\ \hline
		 12 & 8 & 9 \\ \hline
	   \end{tabular}
	\end{center}
	\subsection*{Messung 2}
		\begin{center}
			\begin{tabular}{ l | c | c | r } \hline
			 Interlines & Berechnungszeit & Norm des Fehlers \\ \hline
			 1 & 5 & 6 \\ \hline
			 2 & 5 & 6 \\ \hline
			 4 & 5 & 6 \\ \hline
			 8 & 5 & 6 \\ \hline
			 16 & 5 & 6 \\ \hline
			 32 & 5 & 6 \\ \hline
			 64 & 5 & 6 \\ \hline
			 128 & 5 & 6 \\ \hline
			 256 & 5 & 6 \\ \hline
			 512 & 5 & 6 \\ \hline
			 1024 & 5 & 6 \\ \hline
		   \end{tabular}
		\end{center}
	\subsection*{Messung Scheduling}
		\begin{center}
			\begin{tabular}{ l | c | c | r } \hline
			 Scheduling & Chunk Groesse & Zeit & Norm des Fehlers \\ \hline
		   \end{tabular}
		\end{center}
\end{document}